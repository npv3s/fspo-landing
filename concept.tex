\documentclass[14pt]{extarticle}
\usepackage{xltxtra}
\usepackage{polyglossia}
\setmainlanguage{russian}
\setotherlanguage{english}
\newfontfamily{\cyrillicfont}{Times New Roman}
\usepackage{array, makecell, footnote, setspace, anyfontsize, graphicx, multicol}
\usepackage[paperwidth=210mm, paperheight=297mm, voffset=-5.5mm, hoffset=4.5mm, textheight=257mm, textwidth=165mm]{geometry}
\linespread{1.5}

\title{Концепт лэндинга ФСПО 2.0}
\author{Y2231. Петров Н, Шульженко А, Соколов С.}

\begin{document}
  \maketitle
  %Документ в свободной форме … что Вы будите делать? Какова Ваша идея?
  \hspace{1.25em}Главная идея, на которой мы будем акцентировать внимания в нашем лэндинге - непрерывное погружение от устройства транзисторов на физике, через устройство АЛУ и памяти на информатике, через ассемблер на системном программировании, к высокоуровневым языкам (C++/Java).
  По этому поводу сделаем красивую анимацию.
  Так же большое внимание уделим лэндингу в техническом плане, сделаем его без использования шаблонов, возможно позаимствуем только готовые решения для CSS.
  Сделать сайт "лёгким" и быстрым в техническом плане для нас очень важно, т.к. уровень квалификации наших программистов должен позволять им свободно делать свои решения, а не только использовать готовые.

  \hspace{1.25em}Также в образовательной программе "ФСПО 2.0" мы бы предложили разделить последние группы на курсы с упором на определённые стеки технологий:
  \begin{itemize}
    \item Веб-программирование
    \item Фронтенд разработка
    \item Системное программирование
    \item Администрирование БД
    \item Мобильная разработка
    \item Искуственный интелект
  \end{itemize}
  В рамках работы в этих группах студентам будет предложено углублённо изучить эти направления, познакомиться со всем разнообразием инструментов и узнать особенности работы в этих областях.


\end{document}
